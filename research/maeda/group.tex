\begin{refsection}[research/maeda/group.bib]
\nocite{*}
\chapter{HPC Usability Research Team}

\section{Members}

\begin{itemize}
  \item[] Toshiyuki Maeda (Team Leader)
  \item[] Masatomo Hashimoto (Research Scientist)
  \item[] Peter Bryzgalov (Research Scientist)
  \item[] Itaru Kitayama (Technical Staff I)
  \item[] Yoshiki Nishikawa (Visiting Scientist, University of Tokyo)
%%  \item[] Yves Caniou (Visiting Scientist, Université Claude Bernard Lyon 1)
  \item[] Yves Caniou (Visiting Scientist, Universit\'e Claude Bernard Lyon 1)
  \item[] Judit Gimenez (Visiting Scientist, Barcelona Supercomputing Center)
  \item[] Sameer Shende (Visiting Scientist, University of Oregon)
%%  \item[] Fabien Delalondre (Visiting Scientist, École polytechnique fédérale de Lausanne)
%%  \item[] Pramod Kumbhar (Visiting Scientist, École polytechnique fédérale de Lausanne)
  \item[] Fabien Delalondre (Visiting Scientist, \'Ecole polytechnique f\'ed\'erale de Lausanne)
  \item[] Pramod Kumbhar (Visiting Scientist, \'Ecole polytechnique f\'ed\'erale de Lausanne)
  \item[] Tatsuya Abe (Visiting Scientist, Chiba Institute of Technology)
  \item[] Sachiko Kikumoto (Assistant)
  \item[] Yumeno Kusuhara (Assistant)
\end{itemize}

\section{Research Activities}

The mission of the HPC Usability Research Team is to research and
develop a framework and its theories/technologies for liberating
large-scale HPC (high-performance computing) to end-users and
developers. In order to achieve the goal, we conduct research in the
following three fields:

\subsection{Computing portal}

In a conventional HPC usage scenario, users live in a closed
world. In other words, users have to play roles of software developers,
service providers, data suppliers, and end users. Therefore, a very
limited number of skilled HPC elites can enjoy the power of HPC, while
the general public sometimes gives a suspicious look to the benefit of
HPC. In order to address the problem, we are designing and
implementing a computing portal framework that lowers the threshold
for using, providing, and aggregating computing/data services on HPC
systems, and liberates the power of HPC to the public.

\subsection{Virtualization}

Virtualization is a technology for realizing virtual computers on real
(physical) computers. One big problem of the above mentioned computer
portal that can be used by wide range of users simultaneously is how
to ensure safety, security, and fairness among multiple users and
computing/data service providers. In order to solve the problem, we
plan to utilize the virtualization technology because virtual
computers are isolated from each other, thus it is easier to ensure
safety and security. Moreover, resource allocation can be more
flexible than the conventional job scheduling because resource can be
allocated in a fine-grained and dynamic way. We also study lightweight
virtualization techniques for realizing virtual large-scale HPC for
test, debug, and verification of computing/data services.

\subsection{Program analysis/verification}

Program analysis/verification is a technology that tries to prove
certain properties of programs by analyzing them. By utilizing
software verification techniques, we can prove that a program does not
contain a certain kind of bug. For example, the byte-code verification
of Java VM ensures memory safety of programs. That is, programs that
pass the verification never perform illegal memory operations at
runtime. Another big problem of the above mentioned computing portal
framework is that one computing service can be consists of multiple
computing services that are provided by different
providers. Therefore, if a bug or malicious attack code is contained
in one of the computing services, it may affect the whole computing
service (or the entire portal system). In order to address the
problem, we plan to research and develop software verification
technologies for large-scale parallel programs. In addition, we also
plan to research and develop a performance analysis and tuning
technology based on source code modification history.

\section{Research Results and Achievements}

\subsection{Design and Implementation of a Computing Portal Framework for HPC}

In FY2015, we enhanced the computing portal framework developed in
FY2012-FY2014 so that the users of the K computer are able to build
and publish their own computing portal with their own authority and
computing resources on the K computer. More specifically, we modified
the computing portal framework so that it is able to directly
communicate with the login nodes of the K computer. In addition, the
framework is able to generate SSH public keys (and hide their
corresponding SSH private keys inside) so that the users of the K
computer are able to associate the generate public keys to their
accounts. Thus the users are able to launch their own computing portal
by installing/deploying the framework on their servers and registering
the SSH public keys generated by the framework to their K accounts.

In FY2016, we will continue to develop the computing portal
framework. Especially, we will create a VM image in which the
framework is installed and ready-to-run so that the users of the K
computer is able to launch their own computing portal with the
framework by simply deploying the VM image to, e.g., AWS. In addition,
we will also investigate to develop a Docker
(\url{https://www.docker.com}) container image.

\subsection{Virtualization Techniques}\locallabel{sec:lightweightvirtualization}

\subsubsection{Lightweight Virtualization for Testing/Debugging Parallel Programs}

Writing a program which makes full use of massively parallel HPC
environments (e.g., the K computer) is extremely difficult because
debugging parallel programs is extremely difficult because of inherent
non-determinacy of parallel programs and hard-to-reproduce
bugs. Moreover, writing massively parallel programs also tend to
suffer from a performance problem. For example, even if a program
scales well on a PC cluster system whose size is small-to-moderate,
the program may not scale on massively parallel HPC systems. Even
worse, the performance may severely degrade and will be worse than on
a small PC cluster system or even a single PC. Actually, this is not
uncommon and the reason is that communication costs between computing
nodes may largely vary and sometimes incurs unacceptable heavy
overheads.

In order to address the abovementioned problem, we have been
developing a lightweight network virtualization system for
testing/debugging programs for massively parallel programs without
actually using real massively parallel HPC environments. With our
system, users can run several hundreds of virtual computing nodes on a
single physical computing node.

There are two key ideas in our system: library-hooking and
decentralized management of routing information. Library-hooking is a
kind of virtualization technology which intercepts function calls for
system operations, and modify their parameters and/or return values in
order to trick the programs as if they run on in isolated multiple
computing nodes, even though they run on a single physical computing
node. More specifically, in our lightweight virtualization system, we
mainly hook network related operations (and some file I/O) from user
programs. One benefit of the library-hooking approach is that it
introduces little overheads to program execution (compared to other
virtualization techniques, e.g., CPU level virtualization, OS level
virtualization, and so on) because it can be achieved by user-level
operations and requires no interaction with OS.

When implementing a lightweight network virtualization system,
decentralized management of routing information is necessary in order
to avoid maintaining routing information in a single or a few physical
nodes. Our lightweight virtualization system has to manage routing
information by itself because it virtualizes network environments. If
the routing information is managed in a single physical node, all the
other physical nodes have to ask the single node in order to correctly
route network packets from one virtual node to another. Therefore,
when the numbers of virtual nodes and physical nodes are huge, the
single node will become a performance bottleneck and severely degrade
the overall performance of our lightweight virtualization system. 


More specifically, our lightweight virtualization system statically
distributes the information which virtual node runs on which physical
node before executing user programs on virtual computing nodes. In
ordinary HPC systems, it is uncommon that computing nodes are directly
allocated during job execution. In addition, in order to virtualize
port numbers, our lightweight virtualization system let physical nodes
exchange the information about virtualized port numbers when one
virtual node on one physical node communicates with another virtual
node on another physical node.

Based on the abovementioned approach, in FY2012-FY2014, we implemented
the prototype of our lightweight virtualization system and it
successfully ran on the K computer under a restricted
environment. More specifically, it successfully ran 20000 virtual
computing nodes on 1000 physical computing nodes.  In theory, however,
it must be able to run more virtual computing nodes on a single
physical computing node and run on more physical computing nodes, but
this was impossible because the K computer restricts the number of
user processes on a physical computing node and the operating system
kernel of the computing nodes of the K computer has a serious fault
which is related to memory management.

In order to avoid the abovementioned restriction of the K computer and
run more number of virtual computing nodes, in FY2015, we implemented
workarounds for the prototype of our lightweight virtualization
system. More specifically, we refactored our prototype implementation
and the amount of memory usage were reduced to about 20 \% of the
original one.  We also modified the prototype implementation in order
to keep it from accessing the /etc/hosts file excessively when
establishing connections, by directly passing IP addresses to
mpirun. This modification addressed the problem of memory imbalance
among physical computing nodes. In addition, we upgraded the version
of OpenMPI in order to reduce the amount of memory usage.

In FY2016, we will continue the development of our system and evaluate
it with more numbers of virtual computing nodes and physical computing
nodes. In addition, we plan to study an approach of tricking
performance profiling tools so that they feel as if they run on real
computing nodes and emit profiling data which represents
characteristics of real massively-parallel computing environments.

\subsubsection{Container Technologies for HPC}

Container technologies are a kind of lightweight virtualization
technology. Although they tend to be less efficient than the
library-hooking approach described in the previous section
(Sec.~\localref{sec:lightweightvirtualization}), they provide more
complete image of virtual execution environments. For example, Docker
(\url{https://www.docker.com}) provides multiple isolated virtual
Linux execution environments on a host Linux system. Because Docker is
built and depends on several functionalities provided by the Linux
kernel, it is not able to host non-Linux virtual execution
environments unlike full-virtualization technologies (e.g., KVM, QEMU,
and so on), but far more efficient than them.

One big problem of the current typical HPC systems compared to today’s
so-called cloud services from viewpoint of software
developers/publishers is that the HPC systems are less flexible and/or
responsive. For example, they are not allowed to install and/or modify
system/middleware programs in the HPC systems, while the cloud
services provide fully-virtualized environments to them and they can
freely modify the environments. In addition, the typical HPC systems
are operated with conventional batch schedulers and it sometimes takes
time to launch jobs, while the cloud services launch virtual execution
environments instantly when requested by them.

The reason why the conventional HPC systems are less flexible and/or
responsive is that their primary purpose is to compute scientific
applications efficiently as much as possible, thus the overheads that
may be introduced by utilizing full virtualization technologies are
unacceptable.

On the other hand, as described above, the recent advance in the
container technologies achieves very small overheads yet provides
sufficiently flexible virtual execution environments, thus we predict
that the container technologies will play important role in
forthcoming HPC usage.

In FY2015, we utilized Docker to improve the usability of K-scope,
which is a Fortran source code analysis tool developed by Software
Development Team of AICS.  More specifically, we created a Docker
container image in which K-scope is installed so that users are able
to use K-scope without manually installing it. In addition, we also
extended K-scope so that users are able to analyze their programs
seamlessly on the remote server without modifying their source code
and/or build scripts (please note that, in FY2014, we implemented the
extension of K-scope so that users are able to analyze their programs
on the remote server without installing the analysis engine (more
specifically, XcalableMP, which is developed of Programming
Environment Research Team of AICS), but users may have to modify their
source code and/or build scripts because the paths on the local
machine of users and the remove server may not match).

\subsection{Program Verification and Analysis}

\subsubsection{Memory Consistency Model-Aware Program Verification}

A memory consistency model is a formal model that specifies the
behavior of the memory that is shared and simultaneously accessed by
multiple threads and/or processes. Under the recent multicore CPU
architectures and shared memory multithread/distributed programming
languages (e.g., Java, C++, UPC, Coarray Fortran, and so on), the
shared memory sometimes behaves in an unexpected way because they
adopt relaxed memory consistency models. For example, under some
relaxed memory consistency models, the effects of the memory
operations performed sequentially by one thread (e.g., A $\rightarrow$
B) may be observed in a different order by the other threads (e.g., B
$\rightarrow$ A). Moreover, the threads may not agree on the orders of
the effects of the memory operations (e.g., one thread observes A
$\rightarrow$ B, while the other observes B $\rightarrow$ A, and so
on) they observe. The reason why the recent CPUs and shared memory
languages adopt relaxed memory consistency models is that enforcing
sequential (non-relaxed) memory consistency incurs huge
synchronization overheads among a large number of the threads/nodes
that share a single address memory space.

From the viewpoint of program verification, there are two problems in
handling relaxed memory consistency models. First problem is that the
conventional program verification does not consider relaxed memory
consistency models. Thus, they cannot be applied directly to relaxed
memory consistency models because they may yield false results. Second
problem is that there exist various kinds of relaxed memory
consistency models and each CPU architecture/each programming language
adopts different memory consistency models. Thus, it is very tedious
to define and implement program verification for each CPU and
programming languages of relaxed memory consistency models.

To address these problems, we have been studying three
approaches. First approach is to define a new formal system which is
able to represent various relaxed memory consistency models. More
specifically, we define a very relaxed memory consistency model as a
base model. Then, we define various memory consistency models as
additional axioms on the base model. In fact, we are able to define a
broad range of memory consistency models from CPUs to shared-memory
programming languages (e.g, Intel64, Itanium, UPC, Coarray Fortran,
and so on), in the single formal system.

Second approach is to design and implement a model checker that
supports various relaxed memory consistency models based on the formal
model of the abovementioned first approach. More specifically, we
define a non-deterministic state transition system with execution
traces where each execution trace represents a possible permutation of
instruction executions. Roughly speaking, given a target program, our
model checker explores all the reachable states in the
non-deterministic transition system of the target problem for all the
possible execution traces (that is, permutations of instructions). In
our model checker, memory consistency models can be defined as
constraint rules on execution traces. For example, the sequential
consistency model can be defined as a constraint which allows no
permutation on the execution traces. With our model checker, we were
able to verify the examples programs of the specification manuals of
the memory consistency models of Itanium and UPC.

Third approach is to define a program logic for a shared-memory
parallel process calculus under a relaxed memory consistency
model. More specifically, we define an operational semantics for the
process calculus, and then define a sound (and relatively-complete)
logic to the semantics. There are two key ideas in our program
logic. First idea is that a program is translated into a dependence
graph among instructions in the program, and the operational semantics
and the logic are defined in terms of the dependence graph. One
advantage of handling dependence graphs is that while loops, branch
statements, and parallel composition of processes can be handled in a
uniform way. In addition, another advantage is that multiple memory
consistency models can be handled by adopting different translation
approaches for each memory consistency model. Second idea is that we
introduce auxiliary variables in the operational semantics that
temporarily buffer the effects of memory operations.

In FY2015, we further improved the implementation of McSPIN (developed
from FY2013), and studied the memory consistency model of the
programming language Chapel by request from a research developer of
Chapel. Moreover, we also studied several memory management algorithms
on various memory consistency models with external researchers.

\subsubsection{Evidence-Based Performance Tuning}

To get the maximum of HPC systems, it is inevitable to optimize the
performance of applications. However, performance tuning for massively
parallel HPC systems is very difficult because it is not apparent how
to improve programs except for highly skilled programmers. In
addition, generally speaking, modifying correctly working programs is
a bothering task from the viewpoint of developers. Thus, performance
tuning requires experienced craftsmanship, and relies on intuition and
experience.

In order to address the problem, we have been working on
evidence-based performance tuning. More specifically, we store the
results of performance profiling in a database where the results are
associated with source code modification history. With the database,
developers are able to know, for example, what kinds of optimization
were applied in the past, what kinds of optimization are effective for
improving a certain performance profiling parameter, and so on.

In FY2015, we tried to increase the number of tuning cases in order to
conduct detailed evaluation and improve accuracy of the analysis, but
it turned out that it is hard to collect data directly because we
could not find any researcher/developers who have such the data in and
out of AICS. To work around the problem, we studied an approach of
predicting performance of programs only from their source code
modification history. In fact, we analyzed several thousands of
Fortran projects registered in GitHub (\url{https://github.com}).

In FY2016, we will continue to try to increase the number of tuning
cases in order to conduct detailed evaluation and improve accuracy of
the analysis. More specifically, we will manually develop a training
set by using the results of analysis of the Fortran projects obtained
in FY2015, in cooperation with the two members of Software Development
Team of AICS.

\subsubsection{Python-Based Aggregation of Multiple Software for HPC}

In the world of HPC, programs are usually written in somewhat
old-fashioned programming languages such as Fortran/C/C++ for
historical reasons, thus writing programs for HPC is painful because
we cannot use useful features of modern sophisticated programming
languages. On the other hand, it is not realistic so far to write a
whole program in modern programming languages because of performance
problems.

In order to address the problem and achieve both productivity of
program development and performance of program execution, we studied
an approach of using Python for writing HPC applications. More
specifically, we write non-performance critical large part of a
program in Python, and performance critical small part in
Fortran/C/C++. The reason why we choose Python is that Python provides
a rich set of foreign language interfaces. For example, Fortran
programs can be interfaced with f2py (NumPy:
\url{http://www.numpy.org/}), C programs can be interfaced with ctypes
and Cython, and C++ programs can be interfaced with Boost.Python and
Cython.

In FY2013, we modified EigenExa (a high performance Eigen-solver
developed by the Large-scale Parallel Numerical Computing Technology
research team of AICS) so that it can be used as a shared library and
a Python module (these modifications were feedbacked to the
upstream). In addition, integration of Lotus (a quantum chemistry
library developed by Tomomi Shimazaki, the Computational Molecular
Science research team of AICS) and EigenExa were ongoing mainly by
Tomomi Shimazaki.

In FY2014 and FY2015, in collaboration with Tomomi Shimazaki, a
non-performance critical large part of Lotus were refactored and
written in Python. With the refactoring, we were able to utilize
various existing libraries (e.g., EigenExa:
\url{http://www.aics.riken.jp/labs/lpnctrt/EigenExa\_e.html}, SMASH:
\url{http://smash-qc.sourceforge.net/}, ASE:
\url{https://wiki.fysik.dtu.dk/ase/}, etc.) in Lotus, and demonstrated
that the features of Lotus can be easily extended. More specifically,
we extended Lotus by request of Yukio Kawashima of the Computational
Chemistry research unit of AICS with only several tens of lines of
code addition.  We further refactored Lotus by using Cython in FY2015.

\subsubsection{Porting Performance Analysis Tools to the K computer}

Because massively parallel supercomputers, such as the K computer, are
very different from single computer systems or small size cluster
systems, simply porting existing applications to the K computer
typically does not work due to performance problems (many existing
conventional applications do not consider massively-parallel
systems). Therefore, performance profiling is necessary to understand
the behaviors of applications on massively parallel systems and tune
the applications.

To address the problem, we are porting/deploying existing performance
analysis tools to the K computer, in cooperation with external
research institutes. In FY2014, we ported Scalasca
(\url{http://www.scalasca.org/}) and Extrae
(\url{https://www.bsc.es/computer-sciences/extrae}) to the K
computer. Scalasca was ported in cooperation with a research team of
Jülich Supercomputing Centre, and Extrae was ported in cooperation
with a research team of Barcelona Supercomputing Center. Using the
ported tools, we actually analyzed the behavior of ABySS, a parallel
genome sequence assembler
(\url{http://www.bcgsc.ca/platform/bioinfo/software/abyss}). We also
analyzed the behavior of SIONlib, a parallel I/O library, in
cooperation with Jülich Supercomputing Centre, and deployed it on the
K computer.

In FY2015, we continued to port/deploy existing performance analysis
tools to the K computer. Especially, we ported Eclipse PTP
(\url{https://eclipse.org/ptp/}), which is an extension framework of
Eclipse (\url{https://eclipse.org/}) for parallel program
development/execution, to the K computer, in cooperation with a
research team of University of Oregon. In addition, we modified and
integrated Extrae and SIONlib so that Extrae is able to use SIONlib
for its I/O processing (this should be useful for handling very large
trace data).

%% Text for research Results and achievements. Journal-artcile~\cite{sample-journal}.
%% Conference-paper~\cite{sample-conference}.
%% Invited-talk~\cite{sample-invited}.

%% For cross referencing, use \verb|\locallabel| and \verb|\localref| to avoid conflicting names defined by other groups. For example, a figure can be referenced as Figure~\localref{fig:sample-label1}.

%% \begin{figure}
%% \centering
%%   \includegraphics[width=0.5\textwidth,keepaspectratio,natwidth=193,natheight=40]
%%   {sample_division/sample_group/test1.png}
%%   \caption{Caption for a sample figure}
%%   \locallabel{fig:sample-label1}
%% \end{figure}

\section{Schedule and Future Plan}

In FY2016, we will continue to develop the computing portal
framework. Especially, we will create a VM image in which the
framework is installed and ready-to-run so that the users of the K
computer is able to launch their own computing portal with the
framework by simply deploying the VM image to, e.g., AWS. In addition,
we will also investigate to develop a Docker
(\url{https://www.docker.com}) container image.

Regarding the virtualization technologies, we will continue the
development of our lightweight network virtualization system and
evaluate it with more numbers of virtual computing nodes and physical
computing nodes. In addition, we plan to study an approach of tricking
performance profiling tools so that they feel as if they run on real
computing nodes and emit profiling data which represents
characteristics of real massively-parallel computing environments.

Regarding the program verification and analysis, we will pursue our
evidence-based performance tuning approach. More specifically, we will
continue to try to increase the number of tuning cases in order to
conduct detailed evaluation and improve accuracy of the analysis. More
specifically, we will manually develop a training set by using the
results of analysis of the Fortran projects obtained in FY2015, in
cooperation with the two members of Software Development Team of AICS.

In addition to the above mentioned individual research topics, we also
design/implement integration of the research results of the
virtualization technologies and the software verification with the
computing portal.

%%% DO NOT EDIT BELOW

\section{Publications}

%\printbibliography[keyword=journal, heading=subbibliography, title={Journal Articles}, prefixnumbers={1-}, resetnumbers=true]
%\printbibliography[keyword=proceedings, heading=subbibliography, title={Conference Papers}, prefixnumbers={2-}, resetnumbers=true]
%\printbibliography[keyword=invited, heading=subbibliography, title={Invited Talks}, prefixnumbers={3-}, resetnumbers=true]
%\printbibliography[keyword=poster, heading=subbibliography, title={Posters and Presentations}, prefixnumbers={4-}, resetnumbers=true]
%\printbibliography[keyword=deliverable, heading=subbibliography, title={Patents and Deliverables}, prefixnumbers={5-}, resetnumbers=true]

\printbibliography[keyword=journal, heading=subbibliography, title={Journal Articles}, resetnumbers=true]
\printbibliography[keyword=proceedings, heading=subbibliography, title={Conference Papers}]
\printbibliography[keyword=invited, heading=subbibliography, title={Invited Talks}]
\printbibliography[keyword=poster, heading=subbibliography, title={Posters and Presentations}]
\printbibliography[keyword=deliverable, heading=subbibliography, title={Patents and Deliverables}]

\end{refsection}
